\documentclass[a4paper]{article}
\usepackage{geometry}                % See geometry.pdf to learn the layout options. There are lots.
\geometry{letterpaper}                   % ... or a4paper or a5paper or ... 
%\geometry{landscape}                % Activate for for rotated page geometry
%\usepackage[parfill]{parskip}    % Activate to begin paragraphs with an empty line rather than an indent
\usepackage{graphicx}
\usepackage{amssymb, amsmath}
\usepackage{epstopdf}
\usepackage{float}
\DeclareGraphicsRule{.tif}{png}{.png}{`convert #1 `dirname #1`/`basename #1 .tif`.png}

\title{Report II - MT7027 Risk Models and Reserving in Non-Life Insurance}
\author{Mats Holm}
\date{8 March, 2019}                                           % Activate to display a given date or no date

\begin{document}
\maketitle
\section*{Project II}
\subsection*{Objectives}
In this project we analyze the reserving/prediction method Chain Ladder.
\\


Formulas and equations to calculate reserve:
\begin{align*}
	\hat{f_k} = \dfrac{\sum_{i=1}^{n-k} C_{i,k+1}} {\sum_{i=1}^{n-k} C_{i,k}}
\end{align*}

\begin{align*}
	\hat{R_k} = C_{k,I+1-k} \left(  \hat{f}_{I+1-k} \cdots \hat{f}=_{I-1}-1\right)
\end{align*}
The reserve is then obtained from the sum of $\hat{R_k}$ above.

\subsubsection*{Chain ladder reserve}


\begin{table}[ht]
\centering
\resizebox{\textwidth}{!}{\begin{tabular}{rrrrrrrrrrr}
  & 1 & 2 & 3 & 4 & 5 & 6 & 7 & 8 & 9 & 10 \\ 
  \hline
       11 &4870651& 15576562& 23621698& 27584492& 29640139& 30904769& 31471720& 31767699 &31794628 &31822945\\
       12& 4452874 &16840128 &25553209& 30278599& 32793613 &33823186 &34355699& 34674812& 35005933  &     NA\\
       13& 5676422& 19954053& 29824826& 35025209& 38074008& 40163620& 40993337& 41422048 &      NA    &   NA\\
       14& 5107004& 15510311& 24238656 &27951695& 30326269& 31553444& 32475334 &      NA  &     NA &      NA\\
       15 &4223729& 15370111 &23495224& 28442354 &31151976& 32428748&       NA   &    NA   &    NA &      NA\\
       16& 5040483 &14731877& 22494885& 25756503& 27743509 &      NA&       NA &      NA&       NA  &     NA\\
       17 &2634183& 11313367& 16739575& 20461212  &     NA&       NA &      NA  &     NA  &     NA &      NA\\
       18& 4361924& 15313726& 23612486 &      NA  &     NA    &   NA  &     NA  &     NA   &    NA &      NA\\
       19& 3978609& 12655284  &     NA &      NA   &    NA  &     NA  &     NA   &    NA   &    NA &      NA\\
       20 &3870894     &  NA    &   NA  &     NA &      NA &      NA    &   NA  &     NA  &     NA   &    NA\\
   \hline
\end{tabular}}
\end{table}

From ClaimType 2's cumulative claims triangle 
we use CRAN package $ChainLadder$ \cite{CL} to obtain full triangle.
\begin{table}[H]
\centering
\resizebox{\textwidth}{!}{\begin{tabular}{rrrrrrrrrrr}
 & 1 & 2 & 3 & 4 & 5 & 6 & 7 & 8 & 9 & 10 \\ 
  \hline
11&4870651&15576562&23621698&27584492&29640139&30904769&31471720&31767699&31794628&31822945\\
12&4452874&16840128&25553209&30278599&32793613&33823186&34355699&34674812&35005933&35037110\\
13&5676422&19954053&29824826&35025209&38074008&40163620&40993337&41422048&41645266&41682356\\
14&5107004&15510311&24238656&27951695&30326269&31553444&32475334&32792668&32969383&32998747\\
15&4223729&15370111&23495224&28442354&31151976&32428748&33106359&33429859&33610008&33639942\\
16&5040483&14731877&22494885&25756503&27743509&28923183&29527544&29816074&29976748&30003446\\
17&2634183&11313367&16739575&20461212&22178481&23121526&23604659&23835313&23963758&23985101\\
18&4361924&15313726&23612486&27814039&30148416&31430349&32087098&32400638&32575241&32604253\\
19&3978609&12655284&19253617&22679563&24583013&25628301&26163814&26419475&26561846&26585502\\
20&3870894&13169620&20036122&23601305&25582116&26669886&27227163&27493215&27641372&27665990\\
   \hline
\end{tabular}}
\end{table}


\begin{table}[ht]
\centering
\begin{tabular}{rrrrrrr}
      & Latest& Dev.To.Date&   Ultimate&       IBNR&  Mack.S.E& CV(IBNR) \\
  \hline
11& 31,822,945&       1.000& 31,822,945&          0&         0&      NaN\\
12 &35,005,933 &      0.999& 35,037,110&     31,177&    34,075&   1.0930\\
13 &41,422,048   &    0.994& 41,682,356&    260,308&   293,401 &  1.1271\\
14& 32,475,334 &      0.984& 32,998,747&    523,413 &  251,653&   0.4808\\
15& 32,428,748   &    0.964& 33,639,942& 1,211,194   &332,211 &  0.2743\\
16& 27,743,509  &     0.925 &30,003,446&  2,259,937 &  428,485&   0.1896\\
17& 20,461,212   &    0.853& 23,985,101&  3,523,889 &  426,415 &  0.1210\\
18 &23,612,486   &    0.724 &32,604,253&  8,991,767 &  941,676&   0.1047\\
19 &12,655,284  &     0.476& 26,585,502& 13,930,218 &  982,322&   0.0705\\
20&  3,870,894  &     0.140& 27,665,990& 23,795,096 &3,674,742&   0.1544\\
   \hline
\end{tabular}
\end{table}


The reserve and standard error for ClaimType 2 in Table \ref{tab:mack2}
\begin{table}[!ht]
\center
\begin{tabular}{rr}
	& Totals \\ 
\hline
IBNR: & 54,527,000 \\
Mack S:E & 4,235,733\\
\hline
\end{tabular}
\caption{ClaimType 2} \label{tab:mack2}
\end{table}

Descriptive analysis in Figures \ref{fig:dev} and \ref{fig:mack}.
 \begin{figure}[h]
 \center
  \includegraphics[scale=.5]{devperiod.png}
  \caption{Development periods}
  \label{fig:dev}
\end{figure}

 \begin{figure}[H]
 \center
  \includegraphics[scale=.8]{plotmack.png}
  \caption{Residual plots ClaimType 2}
  \label{fig:mack}
\end{figure}


%% Claim type 1
For ClaimType 1 the costs are sparse.
\begin{table}[ht]
\centering
\begin{tabular}{rrrrrrrrr}
  & 1 & 2 & 3 & 4 & 5 & 6 & 7 & 8  \\ 
  \hline
 11&5943206& 3224539& 523777& 219658 &   NA& 29052& 21205&    NA\\
 12&6760918 &2874648 &912359& 127324& 87549&    NA&    NA &   NA\\
 13&5875200 &3165059 &713833& 265618& 58593&  9358& 15436& 22074\\
 14&6773477 &3069895 &755914 &187402 &68646&    NA&    NA&    NA\\
 15&8860401 &3326446 &796889 &292146& 31856&    NA&    NA&    NA\\
 16&8749801 &4044757& 991210& 140534 &88383 &   NA&    NA&    NA\\
 17&7188952 &3073177& 695189& 255516&    NA&    NA&    NA&    NA\\
 18&8547097 &4552700 &889427 &    NA&    NA&    NA &   NA&    NA\\
 19&5861715 &2460613   &  NA &    NA&    NA&    NA&    NA&    NA\\
20& 6176258   &   NA &    NA &    NA&    NA &   NA&    NA&    NA\\
   \hline
\end{tabular}
\end{table}

After some manipulation we use ChainLadder to get full triangle and the reserve in Table \ref{tab:mack1}.
\begin{table}[ht]
\centering
\resizebox{\textwidth}{!}{\begin{tabular}{rrrrrrrrr}
  & 1 & 2 & 3 & 4 & 5 & 6 & 7 & 8  \\ 
  \hline
11&5943206& 9167745& 9691522& 9911180&9911180&9940232&9961437&9983201\\
12&6760918&9635566 &10547925& 10675249& 10762798& 10783479&10803207&10826811\\
13&5875200&9040259&9754092 &10019710& 10078303& 10087661& 10103097& 10125171\\
14&6773477&9843372& 10599286& 10786688& 10855334& 10876193& 10896091& 10919897\\
15&8860401 &12186847& 12983736& 13275882& 13307738 &13333309 &13357702 &13386887\\
16&8749801& 12794558& 13785768& 13926302& 14014685 &14041614 &14067303& 14098039\\
17&7188952 &10262129 &10957318 &11212834& 11267599& 11289250& 11309903& 11334614\\
18&8547097& 13099797 &13989224& 14255042& 14324665& 14352190& 14378447& 14409862\\
19&5861715&8322328&8929702&9099381&9143823&9161393&9178154&9198207\\
   20& 6176258&  9026318&  9685070&  9869102&  9917304&  9936360&  9954538 & 9976288\\
   \hline
\end{tabular}}
\end{table}

\begin{table}[!ht]
\center
\begin{tabular}{rr}
	& Totals \\ 
\hline
IBNR: & 5,531,171 \\
Mack S:E & 529,551\\
\hline
\end{tabular}
\caption{ClaimType 1} \label{tab:mack1}
\end{table}


 \begin{figure}[H]
 \center
  \includegraphics[scale=.5]{plotmack1.png}
  \caption{Residual plot ClaimType 1}
  \label{fig:mack1}
\end{figure}


\subsection*{Ultimate claim iteration}

We start with a full triangle from which we remove 10 years. Then we calculate 10 ultimate claims each one with one more year of information.
In Figure \ref{fig:ultimate} we see how the estimates approaches 'true' ultimate cost, marked by a straight line.
 \begin{figure}[H]
 \center
  \includegraphics[scale=.5]{uclaims.png}
  \caption{Ultimate claims}
  \label{fig:ultimate}
\end{figure}


\subsection*{Brasklapp}
...

\begin{thebibliography}{99}
\bibitem{CL}
   https://CRAN.R-project.org/package=ChainLadder,
  \emph{ChainLadder: Statistical Methods and Models for Claims Reserving in General Insurance}.
\end{thebibliography}

\section*{Appendix}

\begin{verbatim}
---RRproject2Mack1.R-----------------------------------------------------------
## Risk och reserv Projekt 2

#library(ChainLadder)

claims <- read.table("Projekt2_Grupp8.txt", header = TRUE, sep = ";")
summary(claims)
head(claims, n=10)

sub.1 <- subset(claims, is.element(ClaimType, 1),
                select=c(ClaimDay, PaymentDay, ClaimCost))

sub.1$ClaimYear <- sub.1$ClaimDay %/% 366
sub.1$PaymentYear <- sub.1$PaymentDay %/% 366

sub.1.yearly <-
  aggregate(ClaimCost ~ ClaimYear + PaymentYear, data=sub.1, FUN=sum)
sub.1.yearly$Development <-
  sub.1.yearly$PaymentYear - sub.1.yearly$ClaimYear


m1 <- max(sub.1.yearly$ClaimYear) + 1
n1 <- max(sub.1.yearly$Development) + 1
T.1 <- matrix(0,m1,n1)
for (m in 1:m1)
{
  for (n in 1:n1)
  {
    c <- sub.1.yearly$ClaimCost[
                      sub.1.yearly$ClaimYear == (m-1) &
                      sub.1.yearly$Development == (n-1)]
    if (length(c)>0)
      T.1[m,n] <- c
    else
      T.1[m,n] <- NA
  }
}

T.1[11:nrow(T.1),1:8]

# Cumulative triangle
C.1 <- matrix(NA,m1,n1)
for (m in 1:m1-1)
{
  n.values <- which(!is.na(T.1[m,]))
  n.max <- max(2,n.values)
  C.1[m,1] <- T.1[m,1]
  for (n in 2:n.max)
  {
    if (is.element(n,n.values))
      C.1[m,n] <- C.1[m,n-1] + T.1[m,n]
    else
      C.1[m,n] <- C.1[m,n-1]
  }
}
C.1[m1,1] <- T.1[m1,1]

CC.1 <- C.1[11:nrow(C.1),1:8]

triangle.1 <- as.triangle(CC.1)

mack.1 <- MackChainLadder(triangle.1, est.sigma="Mack")
mack.1
mack.1$f
mack.1$FullTriangle
plot(mack.1)

# Totals
# Latest:   108,727,807.00
# Dev:                0.95
# Ultimate: 114,258,977.85
# IBNR:       5,531,170.85
# Mack.S.E      529,550.67
# CV(IBNR):           0.10

---RRproject2Mack2.R-----------------------------------------------------------
## Risk och reserv Projekt 2

library(ChainLadder)
library(xtable)


claims <- read.table("Projekt2_Grupp8.txt", header = TRUE, sep = ";")
summary(claims)
head(claims, n=10)

sub.2 <- subset(claims, is.element(ClaimType, 2),
                select=c(ClaimDay, PaymentDay, ClaimCost))

sub.2$ClaimYear <- sub.2$ClaimDay %/% 366 + 1
sub.2$PaymentYear <- sub.2$PaymentDay %/% 366 + 1

sub.2.yearly <- 
  aggregate(ClaimCost ~ ClaimYear + PaymentYear, data=sub.2, FUN=sum)
sub.2.yearly$Development <-
  sub.2.yearly$PaymentYear - sub.2.yearly$ClaimYear

# Last 10 years
s.2.yearly <- subset(sub.2.yearly, ClaimYear > 10 & Development < 10)
T.2 <- incr2cum(as.triangle(s.2.yearly,
                                   origin="ClaimYear",
                                   dev="Development",
                                   value="ClaimCost"), na.rm = FALSE)

# Chainladder
n <- 10
f <- sapply(1:(n-1),
               function(i){
                 sum(T.2[c(1:(n-i)),i+1])/sum(T.2[c(1:(n-i)),i])
               }
)

sigma.sq <- sapply(1:(n-1),
            function(i){
              sum(T.2[c(1:(n-i)),i]*(T.2[c(1:(n-i)),i+1]/
                                       T.2[c(1:(n-i)),i] -
                                       f[i])^2)/(n-i-1)
            }
)

# Lazy way
mack.2 <- MackChainLadder(T.2, est.sigma="Mack")
mack.2
mack.2$f
mack.2$FullTriangle
plot(mack.2)



# Check
f
#3.402217 1.521390 1.177938 1.083928 1.042521 1.020895 1.009772 1.005389 1.000891
mack.2$f
#3.402217 1.521390 1.177938 1.083928 1.042521 1.020895 1.009772 1.005389 1.000891 1.000000

sqrt(sigma.sq)
#816.946479 99.178128 129.949537 39.178689 50.198601 33.094542 3.973309 35.430556
mack.2$sigma
#816.946479 99.178128 129.949537 39.178689  50.198601 33.094542 3.973309 35.430556 3.973309


# ITERATION
#First 10 years
s.22.yearly <- subset(sub.2.yearly, ClaimYear < 11 &
                        Development < 10)
T.22 <- incr2cum(as.triangle(s.22.yearly,
                                   origin="ClaimYear",
                                   dev="Development",
                                   value="ClaimCost"),
                                   na.rm = FALSE)

ultimate.claims <- rep(0, 9)
U0 <- matrix(T.22,nrow=10,ncol=10)
for (i in 2:10)
{
  tt <-2
  U <- U0
  for (r in i:10)
  {
    # Upper (left) triangle
    U[r,(10+i-r):10] <- NA
  }
  m <- MackChainLadder(as.triangle(U), est.sigma="Mack")
  ultimate.claims[i-1] <- m$FullTriangle[10,10]
}

t <- 1:9
plot(t,ultimate.claims)
lines(t,rep(U0[10,10],9))

\end{verbatim}


\end{document}  