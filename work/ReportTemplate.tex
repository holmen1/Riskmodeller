\documentclass[11pt]{article}
\usepackage{geometry}                % See geometry.pdf to learn the layout options. There are lots.
\geometry{letterpaper}                   % ... or a4paper or a5paper or ... 
%\geometry{landscape}                % Activate for for rotated page geometry
%\usepackage[parfill]{parskip}    % Activate to begin paragraphs with an empty line rather than an indent
\usepackage{graphicx}
\usepackage{amssymb, amsmath}
\usepackage{epstopdf}
\DeclareGraphicsRule{.tif}{png}{.png}{`convert #1 `dirname #1`/`basename #1 .tif`.png}

\title{Report I - MT7027 Risk Models and Reserving in Non-Life Insurance}
\author{Mathias Lindholm nnnnnn-nnnn \and Filip Lindskog nnnnnn-nnnn \and Some One Else 900101-nnnn}
\date{4 March, 2016}                                           % Activate to display a given date or no date

\begin{document}
\maketitle
\section*{Project I}
\subsection*{Objectives}
({\it Instructions: highlight the project objectives})
\\
In this project we consider ...

\subsection*{Mathematical Background}
({\it Instructions: provide a mathematical background by explaining the key concepts used in your analysis})

Formulas and equations without label:
\begin{align*}
	S_t = \sum_{k=1}^{N_t} X_k
\end{align*}
Formulas and equations with label:
\begin{align}\label{eq:N}
	N_t = \sum_{k=1}^{n} I_k
\end{align}
In \eqref{eq:N} we express $N_t$ in terms of $I_1,\dots,I_n$. The indicator variables are defined on page 16 in \cite{Wuthrich-Merz-13}.

\subsubsection*{The empirical distribution}
The empirical distribution of a sample $Z_1, \dots, Z_n$ of independent and identically distributed random variables or vectors is $\ldots$

\subsubsection*{...other relevant topics...}

\subsection*{Results}

({\it Explain your derivation of ... and display and comment on your simulation results, etc. Include tables and figures as needed and refer to them as follows. In Table \ref{tab:sampletab} we see an example of a sample table. In Figure \ref{fig:samplefig1} we see an example of sample graphics with a single plot, whereas in Figure \ref{fig:samplefig2} we see an example with two plots.}) 

\begin{table}[!ht]
\center
\begin{tabular}{rrr|r}
Name & Street & Number & Room\\ 
\hline
Mathias & Roslagsv. & 101 & 324\\
Filip & Roslagsv. & 101 & 330\\
\hline
\end{tabular}
\caption{Sample table. Use informative captions.} \label{tab:sampletab}
\end{table}




\subsection*{Summary}
({\it Summarize your results and state your conclusions. })



\begin{thebibliography}{99}
\bibitem{Dupire-94}
B. Dupire (1994),
Pricing with a smile.
\emph{Risk}, 7, 18-20.
\bibitem{Wuthrich-Merz-13}
  Mario V. W\"{u}thrich and Michael Merz (2013),
  \emph{Financial Modeling, Actuarial Valuation and Solvency in Insurance},
  Springer-Verlag Berlin Heidelberg.
\end{thebibliography}


\section*{Appendix}

\begin{verbatim}
# Ex 1.3

n_val<-1000
runs<-10000
res_mat2<-matrix(0,runs,3)
for(i in (1:runs))
{
	z_val<-rnorm(1)
	z_vec<-rnorm(n_val)
	res_mat2[i,1]<-sum(exp(z_vec))
	res_mat2[i,3]<-sum(exp(0.1*z_val+sqrt(1-0.1^2)*z_vec))
}

par(mfrow=c(1,2))
hist(res_mat2[,1],75,xlab="",ylab="",main="rho=0",xlim=c(min(res_mat2),max(res_mat2)),
col="black")
hist(res_mat2[,3],75,xlab="",ylab="",main="rho=0.1",xlim=c(min(res_mat2),max(res_mat2)),
col="black")
\end{verbatim}



\end{document}  