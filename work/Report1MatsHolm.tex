\documentclass[11pt]{article}
%\usepackage{geometry}                % See geometry.pdf to learn the layout options. There are lots.
\usepackage[a4paper,hmargin=3cm,vmargin=5cm]{geometry}
%\geometry{a4paper}                   % ... or a4paper or a5paper or ... 
%\geometry{landscape}                % Activate for for rotated page geometry
%\usepackage[parfill]{parskip}    % Activate to begin paragraphs with an empty line rather than an indent
\usepackage{graphicx}
\usepackage{amssymb, amsmath}
\usepackage{epstopdf}
\DeclareGraphicsRule{.tif}{png}{.png}{`convert #1 `dirname #1`/`basename #1 .tif`.png}

\title{Report I - MT7027 Risk Models and Reserving in Non-Life Insurance}
\author{Mats Holm nnnnnn-nnnn }
\date{8 March, 2019}                                           % Activate to display a given date or no date

\begin{document}
\maketitle
\section*{Project I}
\subsection*{Objectives}
({\it Instructions: highlight the project objectives})
\\
In this project we analyze claim
arrival times and claim size distributions for two insurance products.
We will model
the distribution of the total claim cost for claim arrivals from today until one year from
today in case of no reinsurance arrangement...

\subsection*{Mathematical Background}
({\it Instructions: provide a mathematical background by explaining the key concepts used in your analysis})

%%% CLAIM ARRIVAL
\subsection*{Claim arrival process}
By labeling each ClaimDay between 1-365 and plot the histogram we see in Figure \ref{fig:samplefig1} we see that both ClaimTypes show a common decrease during 'summer'. This requires an inhomogenuos process. We chose

\begin{align}\label{eq:N}
	N_t  \sim  \mathrm{Pois}(\mu(t))
\end{align}
In \eqref{eq:N} we chose $\mu(t) = \lambda t$. With $\lambda$ for different claim types and season. Found with the help of R $fitdistr$-function.


\begin{table}[!ht]
\center
\begin{tabular}{r|rr}
Type & Winter & Summer \\ 
\hline
1 & $\lambda=$10.9 & $\lambda=$3.06 \\
2 & $\lambda=$6.77 & $\lambda=$2.1\\
\hline
\end{tabular}
\caption{Sample table. Use informative captions.} \label{tab:sampletab}
\end{table}



 \begin{figure}[!ht]
 \center
  \includegraphics[scale=.5]{histyear.png}
  \caption{Summer slowdown}
  \label{fig:samplefig1}
\end{figure}


%%% CLAIM COST
\subsection*{Claim cost}
First,  we see in Figure \ref{fig:samplefig2} we see that both ClaimTypes show no apparent seasonal variation.

ClaimType 1 lognornam
ClaimType 2 weibull
 Figure \ref{fig:samplefig3} we see that both ClaimTypes show no apparent seasonal variation.


 \begin{figure}[!ht]
 \center
  \includegraphics[scale=.5]{meancost.png}
  \caption{Daily meancost}
  \label{fig:samplefig2}
\end{figure}

 \begin{figure}[!ht]
 \center
  \includegraphics[scale=.5]{histfit.png}
  \caption{Daily meancost}
  \label{fig:samplefig3}
\end{figure}

\subsection*{Dependence}

\subsection*{Bivariate total claims cost}

\subsection*{XL covers}

\subsection*{SL covers}

\subsection*{SL cover}






%
%Formulas and equations without label:
%\begin{align*}
%	S_t = \sum_{k=1}^{N_t} X_k
%\end{align*}
%Formulas and equations with label:
%\begin{align}\label{eq:N}
%	N_t = \sum_{k=1}^{n} I_k
%\end{align}
%In \eqref{eq:N} we express $N_t$ in terms of $I_1,\dots,I_n$. The indicator variables are defined on page 16 in \cite{Wuthrich-Merz-13}.

\subsubsection*{The empirical distribution}
The empirical distribution of a sample $Z_1, \dots, Z_n$ of independent and identically distributed random variables or vectors is $\ldots$

\subsubsection*{...other relevant topics...}


\begin{table}[ht]
\centering
\centering
\resizebox{\textwidth}{!}{\begin{tabular}{rrrrrrrrrrr}
  \hline
 & 1 & 2 & 3 & 4 & 5 & 6 & 7 & 8 & 9 & 10 \\ 
  \hline
1 & 4870651 & 4452874 & 5676422 & 5107004 & 4223729 & 5040483 & 2634183 & 4361924 & 3978609 & 3870894 \\ 
  2 & 15576562 & 16840128 & 19954053 & 15510311 & 15370111 & 14731877 & 11313367 & 15313726 & 12655284 &  \\ 
  3 & 23621698 & 25553209 & 29824826 & 24238656 & 23495224 & 22494885 & 16739575 & 23612486 &  &  \\ 
  4 & 27584492 & 30278599 & 35025209 & 27951695 & 28442354 & 25756503 & 20461212 &  &  &  \\ 
  5 & 29640139 & 32793613 & 38074008 & 30326269 & 31151976 & 27743509 &  &  &  &  \\ 
  6 & 30904769 & 33823186 & 40163620 & 31553444 & 32428748 &  &  &  &  &  \\ 
  7 & 31471720 & 34355699 & 40993337 & 32475334 &  &  &  &  &  &  \\ 
  8 & 31767699 & 34674812 & 41422048 &  &  &  &  &  &  &  \\ 
  9 & 31794628 & 35005933 &  &  &  &  &  &  &  &  \\ 
  10 & 31822945 &  &  &  &  &  &  &  &  &  \\ 
   \hline
\end{tabular}}
\end{table}


\subsection*{Results}

%({\it Explain your derivation of ... and display and comment on your simulation results, etc. Include tables and figures as needed and refer to them as follows. In Table \ref{tab:sampletab} we see an example of a sample table. In Figure \ref{fig:samplefig1} we see an example of sample graphics with a single plot, whereas in Figure \ref{fig:samplefig2} we see an example with two plots.}) 

Formulas and equations without label:
\begin{align*}
	\hat{f_k} = \dfrac{\sum_{i=1}^{n-k} C_{i,k+1}} {\sum_{i=1}^{n-k} C_{i,k}}
\end{align*}

\begin{align*}
	\hat{R_k} = C_{k,I+1-k} \left(  \hat{f}_{I+1-k} \cdots \hat{f}=_{I-1}-1\right)
\end{align*}

%\begin{align}\label{eq:N}
%	\hat{\sigma}_k^2 = \dfrac{1}{I-n-1} \sum_{i=1}^{n-k} C_{i,k}\left( \dfrac{ C_{i,k+1}}{ C_{i,k}} -\hat{f_k} \right)^2
%\end{align}
%In \eqref{eq:N} we express $N_t$ in terms of $I_1,\dots,I_n$. The indicator variables are defined on page 16 in \cite{Wuthrich-Merz-13}.

\begin{table}[!ht]
\center
\begin{tabular}{rr}
	& Totals \\ 
\hline
IBNR: & 54,527,000 \\
Mack S:E & 4,235,733\\
\hline
\end{tabular}
\caption{Sample table. Use informative captions.} \label{tab:sampletab}
\end{table}


% \begin{figure}[!ht]
% \center
% % \includegraphics[scale=.5]{ExampleFig.eps}
%  \caption{Sample graphics. Use informative captions}
%  \label{fig:samplefig1}
%\end{figure}


 \begin{figure}[!ht]
 \center
  %\includegraphics[scale=.35]{ExampleFig.eps}  \includegraphics[scale=.35]{ExampleFig.eps}
  \caption{\emph{Left:} Sample graphics 1. \emph{Right:} Sample graphics 2.}
  \label{fig:samplefig2}
\end{figure}


\subsection*{Summary}
({\it Summarize your results and state your conclusions. })



\begin{thebibliography}{99}
\bibitem{Dupire-94}
B. Dupire (1994),
Pricing with a smile.
\emph{Risk}, 7, 18-20.
\bibitem{Wuthrich-Merz-13}
  Mario V. W\"{u}thrich and Michael Merz (2013),
  \emph{Financial Modeling, Actuarial Valuation and Solvency in Insurance},
  Springer-Verlag Berlin Heidelberg.
\end{thebibliography}


\section*{Appendix}

\begin{verbatim}

#library(ChainLadder)
#library(xtable)


claims <- read.table("Projekt2_Grupp8.txt", header = TRUE, sep = ";")
summary(claims)
head(claims, n=10)

sub.2 <- subset(claims, is.element(ClaimType, 2), select=c(ClaimDay, PaymentDay, ClaimCost))

sub.2$ClaimYear <- sub.2$ClaimDay %/% 366
sub.2$PaymentYear <- sub.2$PaymentDay %/% 366

sub.2.yearly <- aggregate(ClaimCost ~ ClaimYear + PaymentYear, data=sub.2, FUN=sum)
sub.2.yearly$Development <- sub.2.yearly$PaymentYear - sub.2.yearly$ClaimYear

#
sub.2.yearly <- subset(sub.2.yearly, ClaimYear > 9)
triangle.2 <- incr2cum(as.triangle(sub.2.yearly,
                                   origin="ClaimYear",
                                   dev="Development",
                                   value="ClaimCost"), na.rm = FALSE)

mack.2 <- MackChainLadder(triangle.2, est.sigma="Mack")
mack.2
mack.2$f
mack.2$FullTriangle
plot(mack.2)


# Chainladder
n <- 10
f <- sapply(1:(n-1),
               function(i){
                 sum(triangle.2[c(1:(n-i)),i+1])/sum(triangle.2[c(1:(n-i)),i])
               }
)

sigma <- sapply(1:(n-1),
            function(i){
              sum(triangle.2[c(1:(n-i)),i]*(triangle.2[c(1:(n-i)),i+1]/triangle.2[c(1:(n-i)),i] - f[i])^2)/(n-i-1)
            }
)



# Ex 1.3

n_val<-1000
runs<-10000
res_mat2<-matrix(0,runs,3)
for(i in (1:runs))
{
	z_val<-rnorm(1)
	z_vec<-rnorm(n_val)
	res_mat2[i,1]<-sum(exp(z_vec))
	res_mat2[i,3]<-sum(exp(0.1*z_val+sqrt(1-0.1^2)*z_vec))
}

par(mfrow=c(1,2))
hist(res_mat2[,1],75,xlab="",ylab="",main="rho=0",xlim=c(min(res_mat2),max(res_mat2)),
col="black")
hist(res_mat2[,3],75,xlab="",ylab="",main="rho=0.1",xlim=c(min(res_mat2),max(res_mat2)),
col="black")
\end{verbatim}


\end{document}  